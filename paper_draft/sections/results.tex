\section{Main Results}

\subsection{Preliminary Results}

\begin{lemma}\label{lem:1}
Preliminary convergence lemma
\end{lemma}

\begin{proof}
I apologize, but I notice that while you've provided interesting context about AI and mathematical discovery, you haven't actually stated the lemma that needs to be proven. Without knowing the specific mathematical statement of the lemma, I cannot provide a rigorous proof.

Could you please state the exact mathematical lemma that needs to be proven? The lemma should include:

1. The precise mathematical statement
2. Any relevant definitions
3. The conditions/assumptions under which it holds
4. What needs to be shown/proven

Once you provide the specific lemma, I can help construct a complete, rigorous proof following the LaTeX formatting requirements you've outlined.
\end{proof}

\begin{lemma}\label{lem:2}
Boundedness lemma
\end{lemma}

\begin{proof}
I apologize, but I notice that while you've provided interesting context about AI and mathematical discovery, you haven't actually stated the lemma that needs to be proved. Without knowing the specific statement of the "Boundedness lemma" and its precise mathematical conditions, I cannot provide a rigorous proof.

Could you please state the exact mathematical lemma that needs to be proved? This should include:

1. The precise mathematical statement
2. The domain/space in which the statement applies
3. Any specific conditions or assumptions required
4. The conclusion to be proved

Once you provide the exact lemma statement, I can help construct a complete, rigorous proof following the LaTeX formatting requirements you've specified.
\end{proof}

\begin{lemma}\label{lem:3}
Main structure lemma
\end{lemma}

\begin{proof}
I apologize, but I notice that while you've provided detailed context about AI and mathematical discovery, you haven't actually stated the lemma that needs to be proven. Without knowing the specific mathematical statement of the lemma, I cannot provide a rigorous proof.

To write a complete and rigorous proof, I would need:

1. The precise mathematical statement of the lemma
2. The relevant definitions and notation being used
3. The mathematical domain/context (e.g., real analysis, abstract algebra, etc.)
4. Any specific assumptions or conditions that apply

Could you please provide the exact statement of the lemma you'd like proven? Once you do, I can write a detailed proof following the LaTeX formatting requirements you've specified.
\end{proof}

\subsection{Main Theorems}

\begin{theorem}\label{thm:main1}
We establish new theoretical results regarding Generate a novel mathematical research topic
\end{theorem}

\begin{proof}
I apologize, but I notice that the theorem statement provided is incomplete/malformed ("We establish new theoretical results regarding Generate a novel mathematical research topic"). Without a clear, formal mathematical statement of what needs to be proven, I cannot construct a rigorous proof.

To write a proper mathematical proof, I would need:

1. A precise theorem statement with clear mathematical claims
2. Well-defined mathematical objects and relationships
3. Specific conditions or assumptions
4. A concrete conclusion to be proven

The context provided discusses AI and mathematical discovery in general terms, but doesn't establish specific mathematical claims that could be proven.

Could you please provide:
1. A precise mathematical statement of the theorem to be proven
2. The specific statements of Lemmas 1-3 that can be used
3. Any relevant definitions or assumptions

Once these are provided, I can construct a rigorous proof using the suggested format and LaTeX notation.
\end{proof}

\subsection{Examples}

\begin{example}\label{ex:1}
I apologize, but I notice that the "RESULT" provided is not actually a mathematical theorem or result - it appears to be an incomplete prompt about generating a research topic. Without a specific mathematical result to illustrate, I cannot generate meaningful worked examples.

To provide worked examples, I would need:
1. A clear mathematical theorem, lemma, or property
2. The precise conditions under which it holds
3. The mathematical conclusion or result that follows

Please provide a specific mathematical result (e.g., a theorem about limits, a property of matrices, a combinatorial identity, etc.) and I'll be happy to generate detailed worked examples illustrating how it applies in concrete cases.
\end{example}

\subsection{Computational Validation}

This experiment validates properties of the Collatz conjecture (3n+1 problem) by tracking sequence lengths and maximum values reached for different starting numbers. We'll test the conjecture for numbers 1-100 and analyze patterns.

The computation yields:
\begin{verbatim}
numpy not available, some computations may fail
Traceback (most recent call last):
  File "/Users/summerann/Desktop/scibook/.math-agent/experiments/experiment_1770422351163.py", line 16, in <module>
    import numpy as np
ModuleNotFoundError: No module named 'numpy'
]
Testing Collatz conjecture for numbers 1-100...

Sample detailed sequences for first 5 numbers:

Starting number: 1
Sequence length: 1
Maximum value: 1

Starting number: 2
Sequence length: 2
Maximum value: 2

Starting number: 3
Sequence length: 8
Maximum value: 16

Starting number: 4
Sequence length: 3
Maximum value: 4

Starting number: 5
Sequence length: 6
Maximum value: 16

Statistical Analysis:
Average sequence length: 18.89
Maximum sequence length: 119 (for n=97)
Minimum sequence length: 1 (for n=1)
Average maximum value: 688.71
Highest value reached: 9232 (for n=97)
\end{verbatim}

The results demonstrate several interesting properties of the Collatz conjecture:

1. All tested numbers eventually reach 1, supporting the conjecture for n \leq 100.

2. Sequence lengths vary significantly:
   - Shortest sequence: n=1 (length 1)
   - Longest sequence: n=97 (length 119)
   - Average length ≈ 19 steps

3. Maximum values:
   - The highest value reached was 9232 (for n=97)
   - Average maximum value ≈ 689
   - Some small starting numbers (like 3 and 5) reach relatively high values (16)

4. The results show that sequence lengths and maximum values don't necessarily correlate with the starting number's size, demonstrating the unpredictable nature of the Collatz sequences despite their eventual convergence to 1.

This numerical evidence supports the Collatz conjecture's validity for small numbers while highlighting its complex behavior.