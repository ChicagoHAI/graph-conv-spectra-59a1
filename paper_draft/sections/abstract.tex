\begin{abstract}
In this paper, we investigate a novel class of geometric-algebraic structures called quantum-persistent homology rings (QPHRs), which arise naturally in the study of topological data analysis with quantum uncertainty. We establish fundamental properties of these rings and prove a key classification theorem. Our main result shows that for any finite-dimensional quantum system $\mathcal{H}$ with dimension $n \geq 2$, the associated QPHR $\mathcal{R}_{\mathcal{H}}$ satisfies the quantum persistence inequality:

\[
\dim(\mathcal{R}_{\mathcal{H}}) \leq \frac{n(n+1)}{2} \cdot \sum_{k=1}^n \beta_k
\]

where $\beta_k$ denotes the $k$-th quantum Betti number. This bound is tight and achieved precisely when the underlying quantum state space exhibits maximal entanglement. Our proof combines techniques from persistent homology, quantum algebra, and functional analysis. We develop three supporting lemmas characterizing the algebraic structure of QPHRs, their filtration properties, and their behavior under quantum operations. These results have important applications in quantum topological data analysis and provide a new framework for studying quantum systems through their persistent topological features.
\end{abstract}