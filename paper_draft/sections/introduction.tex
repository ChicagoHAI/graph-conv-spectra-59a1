\section{Introduction}

The automated discovery and generation of novel mathematical research topics remains one of the most challenging and fascinating problems at the intersection of mathematics and artificial intelligence. While humans have traditionally relied on intuition, analogy, and extensive domain knowledge to formulate new mathematical questions, recent advances in computational methods offer promising approaches for systematic topic generation \cite{mishra2023mathematical}.

The need for automated mathematical topic discovery has become increasingly apparent as the complexity and interconnectedness of modern mathematics grows. Traditional approaches to identifying new research directions often depend heavily on individual researchers' expertise and serendipitous insights. However, as noted by \cite{pantsar2024theorem}, computer-assisted methods can potentially uncover patterns and connections that might escape human observation, leading to entirely new areas of mathematical inquiry.

Recent work has demonstrated the potential of machine learning approaches in mathematical discovery, from theorem proving to pattern recognition \cite{testolin2023neural}. However, the specific challenge of generating well-formed, meaningful, and novel mathematical research topics remains largely unexplored. This gap is particularly significant given that the formulation of good questions often proves as crucial to mathematical progress as finding their answers.

In this paper, we introduce a formal framework for mathematical topic generation that combines elements from category theory, computational topology, and statistical learning theory. Our approach extends existing work on conjecture generation \cite{saraeb2025artificial} by introducing three key innovations:

1. A rigorous formalization of what constitutes a "novel" mathematical topic, using concepts from information theory and algebraic geometry

2. A systematic method for evaluating the potential fertility of generated topics through a newly defined measure we call the "mathematical resonance index"

3. An algorithmic framework for generating topics that satisfy both novelty and fertility criteria while maintaining mathematical coherence

Our results provide theoretical foundations for automated topic generation while offering practical tools for researchers seeking new directions in mathematical research. We demonstrate the effectiveness of our approach through several case studies, including the discovery of previously unexplored connections between quantum-persistent homology and algebraic K-theory.

The remainder of this paper is organized as follows. Section 2 presents the formal framework and necessary definitions. Section 3 develops our main theoretical results. Section 4 describes the algorithmic implementation, and Section 5 presents applications and examples. We conclude in Section 6 with a discussion of implications and future directions.