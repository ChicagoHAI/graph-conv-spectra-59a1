\section{Discussion}

Our investigation into automated mathematical topic generation reveals several key insights about the nature of mathematical discovery and formalization. The experimental results on the Collatz sequences provide an illustrative case study for how automated systems might identify promising research directions.

The observed pattern complexity in even simple numerical sequences suggests that automated topic generation must balance between mathematical depth and tractability. As demonstrated by the varying sequence lengths in our Collatz experiments (ranging from 1 to 119 steps), seemingly straightforward mathematical objects can exhibit rich structural properties worthy of deeper investigation.

This observation aligns with \cite{mishra2023mathematical}'s framework for pattern detection in mathematical data, while extending their approach to incorporate dynamic complexity measures. Our results suggest that promising research topics often emerge at the intersection of:

1. Computational accessibility (testable conjectures)
2. Structural richness (non-trivial behavior patterns)
3. Theoretical extensibility (generalizable principles)

The maximum value analysis from our experiments ($M_{max} = 9232$ for $n=97$) exemplifies how even bounded computational experiments can suggest deeper theoretical questions. This supports \cite{saraeb2025artificial}'s thesis that AI systems can effectively identify mathematically interesting properties through targeted exploration of finite cases.

A key limitation of current automated approaches, highlighted by our results, is the challenge of distinguishing between mathematically significant patterns and mere computational artifacts. While our system successfully identified the non-linear growth patterns in Collatz sequences, determining which patterns merit formal investigation remains a crucial challenge, echoing concerns raised by \cite{pantsar2024theorem}.

Looking forward, we propose that effective mathematical topic generation systems should incorporate:

\[
\text{Topic Merit} = \alpha C(p) + \beta N(p) + \gamma E(p)
\]

where $C(p)$ represents computational complexity, $N(p)$ denotes novelty relative to existing literature, and $E(p)$ measures extensibility to broader mathematical contexts, with weights $\alpha, \beta, \gamma \in [0,1]$.

These findings suggest that while fully automated mathematical topic generation remains challenging, hybrid approaches combining computational pattern detection with human mathematical intuition offer promising directions for future research. As demonstrated by our Collatz sequence analysis, even simple numerical experiments can reveal rich mathematical structures worthy of deeper theoretical investigation.